\section{Общие сведения}

\subsection{Полное наименование системы и её условное обозначение}
\noindent Полное наименование: Электронная система продажи билетов на междугородние
маршруты. \\
\noindent Условное обозначение: ЭСПБ-ММ.

\subsection{Шифр темы или шифр (номер) договора}
\noindent Шифр договора: Договор №12345 от 01.01.2024 года. \\

\subsection{Наименование предприятий (объединений) разработчика и заказчика
(пользователя) системы и их реквизиты}
\begin{itemize}
    \item Разработчик: ООО «Технологии Будущего», ИНН 1234567890, ОГРН
    9876543210, адрес: г. Москва, ул. Примерная, д. 1.
    \item Заказчик (пользователь): АО «Транспортные Решения», ИНН 0987654321,
    ОГРН 1234567891, адрес: г. Санкт-Петербург, ул. Центральная, д. 5.
\end{itemize}

\subsection{Перечень документов, на основании которых создается система, кем и
когда утверждены эти документы}
\begin{itemize}
    \item Договор №12345 от 01.01.2024 года между ООО «Технологии Будущего» и АО
    «Транспортные Решения».
    \item Техническое задание на создание автоматизированной системы,
    утвержденное приказом директора АО «Транспортные Решения» №12 от 15.01.2024
    года.
    \item ГОСТ 34.602-89 «Техническое задание на создание автоматизированной
    системы».
    \item План мероприятий по цифровизации компании, утверждённый приказом
    генерального директора АО «Транспортные Решения» №23 от 01.12.2023 года.
\end{itemize}

\subsection{Плановые сроки начала и окончания работы по созданию системы}
\begin{itemize}
    \item Начало работ: 01 февраля 2024 года.
    \item Окончание работ: 01 декабря 2024 года.
\end{itemize}

\subsection{Сведения об источниках и порядке финансирования работ}
Финансирование работ осуществляется за счёт собственных средств АО «Транспортные
Решения». Оплата производится поэтапно, согласно графику, указанному в договоре
№12345, на основании актов выполненных работ.

\subsection{Порядок оформления и предъявления заказчику результатов работ по созданию системы (её частей), по изготовлению и наладке отдельных средств (технических, программных, информационных) и программно-технических (программно-методических) комплексов системы}
\begin{itemize}
    \item Результаты работы по созданию системы передаются заказчику поэтапно в
    виде актов приёмки выполненных работ и технической документации.
    \item Для каждого этапа разработки предусмотрены промежуточные акты приёмки
    (технического задания, проектирования системы, разработки функционала,
    тестирования и внедрения).
    \item По завершению работы составляется итоговый акт сдачи-приёмки системы в
    эксплуатацию, включающий все необходимые документы (руководство пользователя,
    руководство администратора, отчёты о тестировании и т.д.).
    \item Окончательная передача системы осуществляется после успешного завершения
    испытаний, включая настройку серверов, установку системы на объекте и проведение
    обучения пользователей.
\end{itemize}

\section{Назначение и цели создания (развития) системы}

\subsection{Назначение системы}
Проектируемая система является электронной системой продажи билетов на
междугородние маршруты, предназначенной для автоматизации процесса поиска,
бронирования и оплаты билетов на междугородние автобусные маршруты. Основные
виды автоматизируемой деятельности включают:
\begin{itemize}
    \item Управление продажей билетов на междугородние автобусные маршруты.
    Обработка платежей за поездки, в том числе из-за рубежа.
    Обеспечение бронирования билетов в реальном времени.
    Автоматизация расчета стоимости поездок с учетом пересадок и использования абонементов.
\end{itemize}

\noindent Объекты автоматизации:
\begin{itemize}
    \item Автобусные транспортные компании (компании, организующие автобусные
    рейсы).
    \item Онлайн пользователи, осуществляющие покупку билетов (как обычные
    пассажиры, так и лица с особыми потребностями).
    \item Административный персонал для управления расписанием и модерации
    системы.
\end{itemize}

\subsection{Цели создания системы}
\noindent Основные цели создания системы:
\begin{itemize}
    \item Повышение скорости обслуживания пассажиров: Предоставление
    пользователям возможности самостоятельно находить маршруты и бронировать
    билеты, что снижает нагрузку на кассы и ускоряет процесс продажи билетов.
    \item Расширение функциональности оплаты: Обеспечение возможности оплачивать
    билеты различными способами (банковские карты, электронные кошельки, СБП,
    SMS), включая оплату из-за рубежа.
    \item Повышение доступности для лиц с особыми потребностями:
    Реализация возможности покупки билетов для пользователей с особыми потребностями
    без необходимости посещения кассы.
    \item Снижение затрат на административное обслуживание: Автоматизация процесса
    управления расписанием, продажей билетов и обработки платежей, что позволяет
    снизить расходы на персонал.
\end{itemize}

\noindent Критерии оценки достижения целей:
\begin{itemize}
    \item Скорость обслуживания: Время, затрачиваемое пользователем на поиск и
    покупку билета, не должно превышать 5 минут.
    \item Надежность оплаты: Успешность проведения транзакций должна составлять
    не менее 99\% от общего числа операций.
    \item Доступность системы: Время доступности системы для пользователей
    должно составлять не менее 99.5\% в год.
\end{itemize}

\section{Характеристика объектов автоматизации}

\subsection{Краткие сведения об объекте автоматизации}
Объектом автоматизации является электронная система продажи билетов на
междугородние автобусные маршруты. Основное предназначение системы --
автоматизация процессов поиска, бронирования и оплаты билетов на автобусные
маршруты, как для внутренних, так и для международных пользователей.

Система охватывает несколько категорий пользователей:
\begin{itemize}
    \item Пассажиры -- пользователи, осуществляющие покупку билетов онлайн на
    междугородние автобусные маршруты.
    \item Корпоративные клиенты -- транспортные компании, организующие
    автобусные рейсы и использующие систему для управления продажей билетов и
    расписанием маршрутов.
    \item Администраторы системы -- сотрудники, отвечающие за техническую
    поддержку, модерацию и управление содержимым системы (например, обновление
    расписания маршрутов).
\end{itemize}

\subsection{Условия эксплуатации объекта автоматизации и характеристики окружающей среды}

\noindent Условия эксплуатации системы:
\begin{itemize}
    \item Система будет эксплуатироваться в онлайн-среде с постоянным доступом к
    интернету для обновления данных в реальном времени.
    \item Основные пользователи (пассажиры и транспортные компании) будут
    взаимодействовать с системой через веб-браузеры на компьютерах, планшетах и
    смартфонах.
    \item Администраторы системы будут использовать специализированные
    интерфейсы для управления контентом и технического сопровождения.
\end{itemize}

\noindent Характеристики окружающей среды:
\begin{itemize}
    \item Физическая среда: Система не требует специфических условий
    эксплуатации для конечных пользователей, так как она доступна через
    интернет.
    \item Техническая среда: Система должна работать в условиях переменного
    качества интернет-соединения, обеспечивая стабильность операций при работе
    на низкоскоростных подключениях.
    \item Программное окружение: Система будет интегрирована с бухгалтерскими
    системами транспортных компаний, а также с системами оплаты и банковскими
    шлюзами для обработки транзакций.
\end{itemize}

\section{Требования к системе}

\subsection{Требования к численности и квалификации персонала}

\noindent Численность персонала:
\begin{itemize}
    \item Ожидается, что системой будут пользоваться три категории
    пользователей: пассажиры (рядовые пользователи), корпоративные клиенты
    (транспортные компании), и администраторы системы.
    \item Для управления системой потребуется минимальное количество
    администраторов (2–3 человека), ответственных за техническую поддержку и
    модерацию контента.
\end{itemize}

\noindent Квалификация персонала:
\begin{itemize}
    \item Для пассажиров и корпоративных клиентов специальной подготовки не
    требуется, они должны обладать базовыми навыками работы с
    интернет-сервисами.
    \item Администраторы системы должны обладать навыками работы с
    веб-интерфейсами управления контентом, а также навыками технического
    обслуживания веб-приложений.
\end{itemize}

\noindent Режим работы:
\begin{itemize}
    \item Администраторы системы работают в режиме 24/7, обеспечивая техническую
    поддержку и обновление данных в реальном времени.
\end{itemize}

\subsection{Требования к показателям назначения}

\noindent Степень приспособляемости:
\begin{itemize}
    \item Система должна поддерживать возможность изменения расписаний маршрутов
    и управления процессами продажи билетов в режиме реального времени.
    \item Допускаются обновления и модификации в течение эксплуатации, чтобы
    адаптироваться к изменению требований транспортных компаний и пользователей.
\end{itemize}

\noindent Модернизация:
\begin{itemize}
    \item Система должна поддерживать возможность интеграции с новыми платежными
    системами и платформами бухгалтерского учета.
    \item Предусмотрена возможность увеличения числа поддерживаемых маршрутов и клиентов
    без необходимости полной замены технической инфраструктуры.
\end{itemize}

\noindent Вероятностно-временные характеристики:
\begin{itemize}
    \item Время отклика системы на запросы пользователей не должно превышать 2
    секунд при стандартных условиях эксплуатации.
    \item Максимальная допустимая продолжительность перерывов в работе системы
    при аварийных ситуациях — не более 5 минут.
\end{itemize}

\subsection{Требования к надежности}

\noindent Показатели надежности:
\begin{itemize}
    \item Доступность системы: Не менее 99.9\% в течение года.
    \item Непрерывность работы: Время работы без сбоев должно составлять не
    менее 1000 часов.
\end{itemize}

\noindent Аварийные ситуации:
\begin{itemize}
    \item В случае отказа серверного оборудования или сбоя в сети, система
    должна автоматически переключиться на резервные сервера, обеспечивая
    минимальные перерывы в обслуживании.
\end{itemize}

\noindent Технические и программные средства:
\begin{itemize}
    \item Программное обеспечение должно обеспечивать автоматическое сохранение
    данных при отключении питания и возможность их восстановления при следующем
    запуске системы.
\end{itemize}

\noindent Оценка и контроль:
\begin{itemize}
    \item На всех стадиях создания системы должны проводиться тестирования
    надежности с использованием методов нагрузочного и стресс-тестирования для
    оценки отказоустойчивости.
\end{itemize}

\subsection{Требования по безопасности}

\noindent Физическая безопасность:
\begin{itemize}
    \item Технические средства должны быть защищены от воздействия
    электрического тока и электромагнитных полей.
\end{itemize}

\noindent Информационная безопасность:
\begin{itemize}
    \item Должна быть реализована защита данных от несанкционированного доступа
    с использованием многофакторной аутентификации и шифрования данных.
    Обеспечение безопасного обмена данными между пользователями, транспортными
    компаниями и администраторами системы.
\end{itemize}

\subsection{Требования к эксплуатации, обслуживанию и хранению}

\noindent Условия эксплуатации:
\begin{itemize}
    \item Система должна быть рассчитана на круглосуточную эксплуатацию без
    постоянного обслуживания.
    \item Периодичность обслуживания серверного оборудования -- один раз в год.
\end{itemize}

\noindent Площади для размещения:
\begin{itemize}
    \item Требуемая площадь для серверного оборудования -- не менее 10
    $\text{м}^2$, с доступом к сетям энергоснабжения 220 В и охлаждением.
\end{itemize}

\noindent Квалификация обслуживающего персонала:
\begin{itemize}
    \item Для технического обслуживания системы требуется квалифицированный
    инженер с опытом работы в сетевых технологиях и серверных системах.
\end{itemize}

\subsection{Требования к защите информации от несанкционированного доступа}
\begin{itemize}
    \item Необходимо реализовать многоуровневую систему защиты данных,
    соответствующую требованиям федерального законодательства о защите
    персональных данных.
\end{itemize}

\subsection{Требования по сохранности информации}
\begin{itemize}
    \item При сбоях в системе (включая потерю питания) должна обеспечиваться
    сохранность всех данных и возможность их восстановления в течение 5 минут.
\end{itemize}

\subsection{Требования к средствам защиты от внешних воздействий}

\noindent Защита от радиопомех:
\begin{itemize}
    \item Технические средства системы должны быть защищены от воздействия
    радиочастотных помех.
\end{itemize}

\noindent Устойчивость к внешним воздействиям:
\begin{itemize}
    \item Все компоненты системы должны сохранять работоспособность при
    температуре от 0°C до 40°C и относительной влажности до 90\%.
\end{itemize}

\subsection{Требования к функциям}

\noindent Основные функции:
\begin{itemize}
    \item Поиск маршрутов и покупка билетов.
    \item Оплата билетов онлайн с различными способами оплаты.
    \item Автоматическая отправка уведомлений о поездке пользователям.
\end{itemize}

\noindent Требования к качеству реализации:
\begin{itemize}
    \item Время отклика на запросы не должно превышать 2 секунд.
    \item Все данные о поездках должны обновляться в режиме реального времени.
\end{itemize}

\subsection{Требования к видам обеспечения}

\noindent Программное обеспечение:
\begin{itemize}
    \item Использование веб-технологий для обеспечения кроссплатформенной
    совместимости.
    \item Поддержка всех современных браузеров и мобильных устройств.
\end{itemize}

\noindent Информационное обеспечение:
\begin{itemize}
    \item Обеспечение взаимодействия с системами транспортных компаний и
    интеграция с бухгалтерией.
\end{itemize}

\noindent Техническое обеспечение:
\begin{itemize}
    \item Использование надежных серверов с поддержкой кластеризации для
    повышения отказоустойчивости.
\end{itemize}

\section{Состав и содержание работ по созданию системы}

\subsection{Перечень стадий и этапов работ}
\noindent Предпроектные работы:
\begin{itemize}
    \item Анализ требований и определение основных характеристик системы.
    \item Подготовка Технического задания (ТЗ).
    \item Срок выполнения: 1 месяц.
    \item Исполнители: Заказчик, разработчик.
\end{itemize}

\noindent Техническое проектирование:
\begin{itemize}
    \item Разработка архитектуры системы и детальное описание подсистем.
    \item Определение требований к программному и техническому обеспечению.
    \item Срок выполнения: 2 месяца.
    \item Исполнители: Разработчик.
\end{itemize}

\noindent Разработка программного обеспечения:
\begin{itemize}
    \item Разработка, тестирование и отладка основных модулей системы.
    \item Реализация пользовательских интерфейсов.
    \item Срок выполнения: 3 месяца.
    \item Исполнители: Разработчик.
\end{itemize}

\noindent Интеграция и тестирование:
\begin{itemize}
    \item Интеграция всех компонентов системы и тестирование в условиях реальной эксплуатации.
    \item Исправление ошибок и оптимизация производительности.
    \item Срок выполнения: 1 месяц.
    \item Исполнители: Разработчик.
\end{itemize}

\noindent Ввод в эксплуатацию:
\begin{itemize}
    \item Установка системы у заказчика, обучение персонала.
    \item Настройка системы и проверка ее функционирования.
    \item Срок выполнения: 1 месяц.
    \item Исполнители: Разработчик, заказчик.
\end{itemize}

\subsection{Перечень документов}
По завершении каждой стадии и этапа работ разрабатываются и передаются следующие документы:
\begin{itemize}
    \item По завершении предпроектных работ -- Техническое задание.
    \item По завершении технического проектирования -- Технический проект.
    \item По завершении разработки программного обеспечения -- Программная документация и Руководство пользователя.
    \item По завершении интеграции и тестирования -- Отчет о тестировании, Акт приемки системы.
\end{itemize}

\subsection{Вид и порядок проведения экспертизы технической документации}
Экспертиза проводится на следующих этапах:
\begin{itemize}
    \item Техническое задание -- экспертиза заказчиком и независимой организацией-экспертом.
    \item Технический проект -- экспертиза разработчиком и сторонней экспертной организацией.
    \item Программное обеспечение -- независимая проверка на соответствие требованиям и тестирование на отказоустойчивость.
\end{itemize}

\subsection{Программа обеспечения надежности}
Программа включает работы по обеспечению стабильной работы системы в условиях
интенсивной нагрузки и защиты от отказов. Работы проводятся на этапе
тестирования и вводятся в эксплуатацию.

\subsection{Ответственность за выполнение работ}
\begin{itemize}
    \item Заказчик -- ответственность за утверждение Технического задания и согласование результатов работ на каждом этапе.
    \item Разработчик -- ответственность за разработку и реализацию системы, ее тестирование и ввод в эксплуатацию.s
\end{itemize}

\section{Порядок контроля и приемки системы}

Данный раздел определяет порядок контроля и приемки системы, описывая виды испытаний, общие требования к приемке работ, а также статус приемочной комиссии.

\subsection{Виды, состав, объем и методы испытаний системы}
\noindent Виды испытаний:
\begin{itemize}
    \item Предварительные испытания: проводятся для проверки отдельных
    компонентов системы на соответствие техническим требованиям. Эти испытания
    включают функциональные и нагрузочные тесты, а также тестирование
    отказоустойчивости.
    \item Интеграционные испытания: проводятся после завершения этапа интеграции
    для проверки совместной работы всех компонентов системы.
    \item Приемочные испытания: включают полное тестирование системы в условиях
    реальной эксплуатации. Включает проверку всех функций и задач системы, а
    также соответствие заявленным требованиям производительности, надежности и
    безопасности.
\end{itemize}

\noindent Состав и объем испытаний:
\begin{itemize}
    \item Функциональные испытания: проверка всех заявленных функций системы и
    их соответствие Техническому заданию.
    \item Нагрузочные испытания: проверка системы на предельных нагрузках для
    определения ее производительности и стабильности.
    \item Испытания на отказоустойчивость: имитация аварийных ситуаций для
    проверки способности системы к восстановлению после отказов.
    \item Испытания безопасности: проверка соблюдения норм по защите информации
    и безопасности эксплуатации.
\end{itemize}

\noindent Методы испытаний:
\begin{itemize}
    \item Автоматизированные тесты для проверки функциональности и производительности.
    \item Имитация реальных сценариев работы для оценки поведения системы в условиях эксплуатации.
    \item Тестирование на соответствие ГОСТ и другим применимым нормативно-техническим документам.
\end{itemize}

\subsection{Общие требования к приемке работ по стадиям}
\noindent Участники приемки:
\begin{itemize}
    \item В приемке работ участвуют представители заказчика, разработчика, а также независимые эксперты, назначенные заказчиком или ведомственными органами.
\end{itemize}

\noindent Порядок приемки работ:
\begin{itemize}
    \item По завершении предпроектных работ -- согласование и утверждение
    Технического задания (ТЗ).
    \item По завершении технического проектирования -- проверка технического проекта на
    соответствие требованиям, утвержденным в ТЗ, и согласование проектной
    документации.
    \item По завершении разработки программного обеспечения -- тестирование и
    приемка программных модулей, проверка их соответствия требованиям
    производительности, надежности и безопасности.
    \item По завершении интеграционных испытаний -- проведение испытаний на
    совместимость компонентов системы и их готовность к эксплуатационным
    испытаниям.
    \item По завершении приемочных испытаний -- комплексная проверка системы и
    утверждение результатов для окончательной сдачи проекта заказчику.
\end{itemize}

\noindent Порядок согласования и утверждения приемочной документации:
\begin{itemize}
    \item По итогам каждого этапа разрабатывается соответствующая документация,
    включая акты испытаний и отчеты о проверке. Эти документы согласуются с
    участниками приемочного процесса и утверждаются заказчиком.
\end{itemize}

\subsection{Статус приемочной комиссии}
Приемочная комиссия будет межведомственной, если проект включает участие
нескольких организаций или ведомств, или государственной, если система требует
проверки и приемки на государственном уровне.

\noindent В состав комиссии входят:
\begin{itemize}
    \item Представители заказчика.
    \item Представители разработчика.
    \item Независимые эксперты.
    \item Представители государственных органов (при необходимости).
\end{itemize}

Комиссия отвечает за финальную оценку системы и принятие решения о ее соответствии установленным требованиям для ввода в эксплуатацию.

\section{Требования к составу и содержанию работ по подготовке объекта
автоматизации к вводу системы в действие}

Этот раздел описывает мероприятия, которые необходимо провести для подготовки объекта автоматизации к успешному вводу автоматизированной системы (АС) в эксплуатацию. Включаются ключевые действия по подготовке объектов и создание условий, необходимых для функционирования системы, в соответствии с требованиями, установленными в техническом задании (ТЗ).

\subsection{Приведение информации к виду, пригодному для обработки с помощью ЭВМ}
Для корректной работы автоматизированной системы, поступающая информация должна быть подготовлена следующим образом:
\begin{itemize}
    \item Формат данных: Данные, поступающие в систему, должны быть структурированы и приведены в соответствии с требованиями информационного и лингвистического обеспечения системы.
    \item Обработка данных: Преобразование имеющихся данных в формат, подходящий для дальнейшей обработки с использованием электронных вычислительных машин (ЭВМ). Например, преобразование бумажных документов в цифровой формат, структурирование баз данных.
    \item Унификация данных: Проверка и корректировка данных для их согласования с классификаторами и стандартами, принятыми в системе, а также устранение дублирующей или некорректной информации.
\end{itemize}

\subsection{Изменения, которые необходимо осуществить в объекте автоматизации}
Для обеспечения совместимости и успешного функционирования системы могут потребоваться следующие изменения в объекте автоматизации:

\begin{itemize}
    \item Техническая модернизация: Обновление или установка дополнительного оборудования, необходимого для работы системы (например, новые серверы, сетевое оборудование, периферийные устройства).
    \item Изменение процессов: Внедрение новых процессов или модификация существующих бизнес-процессов для интеграции с автоматизированной системой. Это может включать пересмотр методов управления, изменения в регламенте работы, автоматизацию отдельных операций.
    \item Физические изменения: Возможное изменение инфраструктуры объекта автоматизации для соответствия требованиям системы (например, модернизация серверных помещений, улучшение электросетей).

\end{itemize}

\subsection{Создание условий для функционирования объекта автоматизации}

Для гарантии того, что система будет соответствовать требованиям ТЗ, необходимо
создать условия, способствующие ее корректной работе:
\begin{itemize}
    \item Организация инфраструктуры: Обеспечение надлежащих условий для работы серверов, рабочих станций, сетевого оборудования, систем хранения данных, а также стабильного энергоснабжения и системы охлаждения.
    \item Обеспечение надежности и безопасности: Внедрение мер по защите информации, резервному копированию и восстановлению данных, предотвращению сбоев, обеспечению кибербезопасности.
    \item Оптимизация рабочих процессов: Регулярное тестирование и настройка системы для обеспечения бесперебойной работы компонентов системы в соответствии с требованиями производительности и надежности.

\end{itemize}

\subsection{Создание необходимых подразделений и служб}
Для успешного функционирования автоматизированной системы должны быть созданы следующие подразделения и службы:
\begin{itemize}
    \item Служба технической поддержки: Подразделение, ответственное за эксплуатацию, техническое обслуживание и ремонт системы.
    \item Отдел информационной безопасности: Группа, ответственная за защиту данных от несанкционированного доступа и других угроз.
    \item Административные и операционные подразделения: Персонал, ответственный за управление системой, её настройку и контроль выполнения операций.
\end{itemize}

\subsection{Сроки и порядок комплектования штатов и обучения персонала}
\begin{itemize}
    \item Подбор персонала: Определение численности и квалификации сотрудников, необходимых для эксплуатации системы, и комплектование штатов в соответствии с этими требованиями.
    \item Обучение персонала: Организация программ подготовки и повышения квалификации сотрудников, которые будут работать с системой. Включает обучение как технических специалистов, так и конечных пользователей системы.
    \item Контроль знаний: Регулярная проверка знаний и навыков персонала для обеспечения высокой квалификации при работе с системой, а также проведение сертификационных экзаменов по окончании обучения.
\end{itemize}

\section{Требования к документированию}

Данный раздел описывает требования к составу, содержанию и форматам документации, которая должна быть разработана в рамках создания автоматизированной системы (АС). Документация должна соответствовать государственным стандартам, нормативно-технической документации (НТД) отрасли заказчика и стандартам по единой системе конструкторской (ЕСКД) и проектной документации (ЕСПД).

\subsection{Перечень подлежащих разработке документов}
В данном пункте приводится согласованный разработчиком и заказчиком перечень документов, которые должны быть созданы в рамках проекта, включая:
\begin{itemize}
    \item Комплекты и виды документов: Все документы должны соответствовать требованиям ГОСТ 34.201 и НТД, действующим в отрасли заказчика. Сюда могут включаться:
    \begin{itemize}
        \Item Техническое задание (ТЗ),
        \Item Технический проект (ТП),
        \Item Рабочая документация (РД),
        \Item Описание алгоритмов и программ,
        \Item Руководства пользователя и администратора,
        \Item Инструкции по эксплуатации.
    \end{itemize}
    \item Документы на машинных носителях: Список документов, которые будут выпускаться в электронном виде на машинных носителях. Эти документы могут включать программное обеспечение, технические инструкции, отчеты об испытаниях и т.д.
\end{itemize}

\subsection{Документирование элементов межотраслевого применения}
Для компонентов и элементов системы, которые могут использоваться в других отраслях или системах, документация должна быть разработана в соответствии с требованиями:
\begin{itemize}
    \item ЕСКД и ЕСПД: Документы должны быть оформлены в соответствии с правилами единой системы конструкторской документации и единой системы проектной документации, чтобы обеспечить стандартизацию и унификацию.
    \item Универсальные технические требования: Для элементов, применяемых в различных системах, документация должна быть разработана так, чтобы обеспечивать их повторное использование и совместимость с другими проектами.
\end{itemize}

\subsection{Дополнительные требования к документированию}
В случаях, когда отсутствуют государственные стандарты, определяющие требования к документированию элементов системы, дополнительно могут быть установлены следующие требования:
\begin{itemize}
    \item Состав и содержание документации: Требования к описанию компонентов системы, которые не подпадают под стандарты ГОСТ или НТД, должны быть согласованы с заказчиком и включать полное описание функций, конструктивных особенностей и технических характеристик.
    \item Формат документов: В отсутствие стандартов разработчик может предложить собственные форматы документов, при условии их согласования с заказчиком. Это может касаться как текстовых, так и графических материалов (чертежи, схемы, диаграммы).
\end{itemize}

\section{Источники разработки}

В данном разделе приводится перечень документов, отчетов и информационных материалов, которые использовались в процессе разработки технического задания (ТЗ) и будут применяться при создании автоматизированной системы (АС). Это могут быть как внутренние, так и внешние материалы, обеспечивающие обоснование требований, целей и функций разрабатываемой системы.

\subsection{Документы и материалы}
В раздел включаются следующие категории источников:
\begin{itemize}
    \item Технико-экономическое обоснование (ТЭО): Документы, содержащие обоснование экономической целесообразности и выгоды создания системы, её влияния на производственные или управленческие процессы, а также предполагаемую окупаемость проекта.

    \item Отчеты о научно-исследовательских работах (НИР): Законченные отчеты о научных исследованиях, проведенных ранее и связанных с проектируемой системой. Они могут включать новые технологии, алгоритмы, математические модели и другие результаты, которые могут быть полезны для разработки.

    \item Информационные материалы о системах-аналоги: Описание отечественных и зарубежных аналогичных систем, которые могут служить примером при проектировании АС. Сюда могут входить описания функций, технические характеристики, успешные решения, а также анализ недостатков, которые нужно учесть при разработке.

    \item Отраслевые нормативы и стандарты: Список нормативных документов (ГОСТы, СНиПы, международные стандарты и т.д.), используемых при проектировании и создании системы.

\end{itemize}

\subsection{Приложения к техническому заданию}
По согласованию между разработчиком и заказчиком в состав технического задания могут быть включены дополнительные приложения, если на этапе разработки были использованы утвержденные методики.

\begin{itemize}
    \item Расчет ожидаемой эффективности системы: Приложение, содержащее расчетные данные, подтверждающие, что система будет соответствовать ожиданиям в отношении эффективности работы. Это может включать улучшения в производительности, сокращение времени на выполнение задач, экономию ресурсов, повышение качества управления и т.п.

    \item Оценка научно-технического уровня системы: Приложение, содержащее оценку новизны применяемых технологий, уровень автоматизации, использование передовых методов и соответствие мировым или отраслевым стандартам.

\end{itemize}

Приложения включаются в ТЗ по согласованию с заказчиком, если это необходимо для дополнительного обоснования эффективности и научно-технического уровня проекта.
