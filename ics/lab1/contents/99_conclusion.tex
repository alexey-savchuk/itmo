В ходе выполнения работы были изучены и применены методики определения
требований к инфокоммуникационной системе. Это позволило сформулировать
пользовательские истории и детализировать функциональные требования для
различных категорий пользователей системы, учитывая их цели и ожидания. Также
была изучена и реализована основа разработки функциональных моделей с
использованием методологии IDEF0. Были построены контекстная диаграмма и
декомпозиция 1-го уровня, что дало возможность визуализировать основные процессы
системы и определить их взаимосвязи с внешними элементами, такими как данные,
механизмы и средства контроля.
