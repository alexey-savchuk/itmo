\subsection{Пользователи системы}

Выделим следующие классы пользователей: конечные пользователи, корпоративные
клиенты, администраторы.

\subsubsection*{Конечные пользователи}

Это пользователи, которые непосредственно используют систему для поиска и
покупки билетов на междугородние автобусы. Класс содержит единственный тип
\textit{пассажиры междугородних автобусных маршрутов}, который включает
следующие роли.

\null

\noindent\textit{Пользователь-гость}

\noindent цели:
\begin{itemize}
    \item быстро найти доступные маршруты и узнать стоимость билетов без
    необходимости регистрации;
    \item получить общую информацию о поездках и доступных вариантах.
\end{itemize}

\noindent ожидания:
\begin{itemize}
    \item удобный и быстрый интерфейс для поиска маршрутов;
    \item минимум шагов для поиска информации;
    \item возможность легко зарегистрироваться для покупки билета.
\end{itemize}

\null

\noindent\textit{Рядовой пользователь}

\noindent цели:
\begin{itemize}
    \item купить билет на нужный маршрут быстро и удобно;
    \item иметь возможность сохранять свои данные для упрощения
    последующих покупок.
\end{itemize}

\noindent ожидания:
\begin{itemize}
    \item персонализированный опыт: сохраненные маршруты, пассажиры,
    история поездок;
    \item удобная система оплаты и быстрая обработка транзакций;
    \item поддержка разных способов оплаты.
\end{itemize}

\null

\noindent\textit{Пользователь с особыми потребностями}

\noindent цели:
\begin{itemize}
    \item купить билет на нужный маршрут без необходимости посещать
    кассу;
    \item получить доступ к адаптированной системе с улучшенной
    доступностью.
\end{itemize}

\noindent ожидания:
\begin{itemize}
    \item удобный интерфейс для самостоятельной покупки билета онлайн;
    \item настраиваемая цветовая схема и крупные элементы интерфейса;
    \item возможность переключения на более доступные версии сайта;
    \item поддержка специальных возможностей, таких как озвучивание
    текста.
\end{itemize}

\null

\noindent\textit{Иностранный пользователь}

\noindent цели:
\begin{itemize}
    \item легко пользоваться системой на понятном языке;
    \item купить билет на поездку быстро и удобно.
\end{itemize}

\noindent ожидания:
\begin{itemize}
    \item поддержка нескольких языков, включая перевод интерфейса и
    инструкций;
    \item возможность оплаты в валюте или с использованием международных
    платёжных систем.
\end{itemize}

\subsubsection*{Корпоративные клиенты}

Это компании, предоставляющие услуги междугородних автобусных перевозок и
использующие систему для автоматизации учета транзакций. Класс содержит
единственный тип \textit{компании-организаторы междугородних автобусных
перевозок}, который включает единственную роль.

\null

\noindent\textit{Автоматизированная система бухгалтерского учета}

\noindent цели:
\begin{itemize}
    \item получать данные о транзакциях за билеты для автоматической
    обработки;
    \item интегрировать систему оплаты с 1C:Бухгалтерия для облегчения
    учёта.
\end{itemize}

\noindent ожидания:
\begin{itemize}
    \item простая и нативная интеграция с REST API 1C:Бухгалтерия без
    сложных дополнительных настроек;
    \item возможность выбора между выгрузкой данных в режиме реального
    времени или по расписанию;
    \item полная автоматизация процессов бухгалтерского учёта с
    минимальными затратами на интеграцию.
\end{itemize}

\subsubsection*{Администраторы}

Это пользователи, которые управляют и поддерживают
работоспособность системы, а также редактируют контент, связанный с расписаниями
и другими данными. Класс содержит следующие типы и роли:
\begin{itemize}
    \item (тип) технические администраторы -- (роль) системный администратор;
    \item (тип) контент-администраторы -- (роль) модератор расписаний.
\end{itemize}

\null

\noindent\textit{Системный администратор}

\noindent цели:
\begin{itemize}
    \item эффективное управление учетными записями пользователей;
\end{itemize}

\noindent ожидания:
\begin{itemize}
    \item доступ к интерфейсу для создания, редактирования, блокировки и
    удаления учетных записей;
    \item возможность сброса паролей;
\end{itemize}

\null

\noindent\textit{Модератор расписаний}

\noindent цели:
\begin{itemize}
    \item управление расписаниями автобусных маршрутов;
    \item обеспечение актуальности и корректности информации о рейсах.
\end{itemize}

\noindent ожидания:
\begin{itemize}
    \item доступ к интерфейсу для добавления, изменения и удаления
    расписаний маршрутов;
    \item возможность быстро вносить коррективы при изменениях в
    маршрутах или расписаниях.
\end{itemize}
